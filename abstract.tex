%% The following is a directive for TeXShop to indicate the main file
%%!TEX root = diss.tex


%  "it must not exceed 350 words in length" -- https://www.grad.ubc.ca/current-students/dissertation-thesis-preparation/structure-theses-dissertations

\chapter{Abstract}


When compared to conventional inorganic semiconductors, organic semiconductors are lightweight, flexible, and compatible with less expensive high-throughput manufacturing techniques. Applications of organic semiconductors in power generation and light emitting applications have been realized through the development of \ac{OPV} and \ac{OLED} devices. However, to optimize the performance and efficiency of these applications, the molecular orbital energy and the role of the exciton in charge generation and luminescence in organic materials need to be further explored.

In this work, \ac{SPM} techniques including \ac{STM}, \ac{STS}, and \ac{STML} were used to probe the electronic and optical properties of individual organic molecules deposited on insulating NaCl layers on a metallic substrate. Pixel-by-pixel \ac{STS} energetically and spatially resolves molecular orbitals. Concurrent \ac{STML} induces molecular luminescence through electron tunnelling, giving spectral information of the excitons and vibrational modes of the organic molecule on sub-nanometre length scales.

\sloppy The results presented here are the first signals of molecular emission obtained from our microscope, demonstrating the capability of our system in detecting single molecule luminescence.  Conventional organic molecules \ac{PTCDA} and \ac{ZnPc} were studied and the results compared to those presented in the literature. \ac{SPM} was also performed on \ch{F8ZnPc} to explore the effects of fluorination. Our results revealed that electronic and structural changes due to the additional fluorine atoms and interactions with the substrate can affect the energy levels and luminescence of the molecule.

\sloppy Organic donor-acceptor molecules based on the \ac{HMAT} complex were also studied. The effects of various acceptor groups on the energetic gap and spatial distribution of molecular orbitals were explored for different \ac{HMAT} derivatives using \ac{STS}. We demonstrate the effects of gap engineering at the sub-molecular level for this promising class of optoelectronic organic materials.

%but we were unable to correlate the effects to excitonic behaviour due to the instability of the molecules and our system.









% Consider placing version information if you circulate multiple drafts
% \vfill
% \begin{center}
% \begin{sf}
% \fbox{Revision: \today}
% \end{sf}
% \end{center}
