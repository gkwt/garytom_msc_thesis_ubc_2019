%% The following is a directive for TeXShop to indicate the main file
%%!TEX root = diss.tex

\chapter{Introduction}
\label{ch:Introduction}

%%%%%
\section{Motivation}

% why are we interested in organic semiconductors
%   various applications, focusing specifically on OPV and OLED

% what is the interesting physics that we want to probe

The rapid development of silicon-based and inorganic semiconductors, starting from the early 1950s, has since led to the age of modern computing and electronics. In parallel, the first conducting organic polymer, polyacetylene, was synthesized in 1977 \citep{shirakawa1977synthesis,chiang1977electrical}. The discovery of non-insulating organic plastics established a new research area focused on the development of organic semiconductors, and led to the Chemistry Nobel Prize in 2000 being awarded to their discoverers Heeger, MacDiarmid, and Shirakawa. After several decades of research, the class of semiconducting organic molecules has grown substantially, and organic semiconductors have found success in commercial applications such as \acfp{OLED} \citep{zhu2011solution,ying2014white}, organic field effect transistors \citep{torsi2013organic,tatum2018pi}, and \acfp{OPV} \citep{holliday2017recent,espinosa2015solution,gregg2003comparing,lewis2007toward}. 

The major advantage of organic semiconductors, when compared to their inorganic counterparts, is their relative ease of processing. Organic molecules can be thermally deposited or solution processed at relatively low temperatures, allowing for cost-effective device fabrication \citep{yip2012recent,huang2019organic}. Additionally, by drawing from a suite of chemical synthesis techniques, new molecules can be created largely from Earth-abundant elements, and engineered with different functional groups to have the desired optical, electronic, and structural properties. This versatility allows for tunable light absorption and emission, transparency, and mechanical flexibility, which have already been applied to niche commercial products, such as visibly transparent solar cells or foldable displays (\autoref{fig:intro:applications}). 

\begin{figure}[h]
    \centering
    
    \begin{subfigure}[t]{0.43\textwidth}
        \centering
        \includegraphics[width=0.8\textwidth]{pictures/transparent_solar.PNG}
        \caption{}
    \end{subfigure}
    \hspace{0.5cm}
    \begin{subfigure}[t]{.43\textwidth}
        \centering
        \includegraphics[width=\textwidth]{pictures/flexible_opv.jpg}
        \caption{}
    \end{subfigure}
    
    
    \vspace{0.2cm}
    
    \begin{subfigure}[t]{.43\textwidth}
        \centering
        \includegraphics[width=\textwidth]{pictures/folding_oled_fold.PNG}
        \caption{}
    \end{subfigure}
    \hspace{0.5cm}
    \begin{subfigure}[t]{.43\textwidth}
        \centering
        \includegraphics[width=0.95\textwidth]{pictures/biological_oled.PNG}
        \caption{}
    \end{subfigure}
    
    
    \caption[Various applications of organic semiconducting materials. \textbf{(a)} Organic solar cell that absorbs in the infrared range, making it transparent to visible light. \textbf{(b)} Flexible thin film {OPV} that are fabricated in large rolls (Infinity PV). \textbf{(c)} Foldable display made possible by flexible {OLED} technology (Samsung Electronics). \textbf{(d)} Biological imaging application of {OLED} molecules attached to nanoparticles.]{Various applications of organic semiconducting molecules. \textbf{(a)} Organic solar cell that absorbs in the infrared range, making it transparent to visible light (\citeauthor{zhao2014near} \citep{zhao2014near}). \textbf{(b)} Flexible thin film {OPV} that are fabricated in large rolls (Infinity PV \citep{jacobyInfinityPV}). \textbf{(c)} Foldable display made possible by flexible {OLED} technology (Samsung Electronics \citep{galaxyFold}). \textbf{(d)} Biological imaging application of {OLED} molecules attached to nanoparticles (\citeauthor{crossley2017post} \citep{crossley2017post}).  }
    \label{fig:intro:applications}
\end{figure}

Organic materials possess low electronic screening, a result of their intrinsically low dielectric constant (or low electric susceptibility), which presents challenges in optimizing device performance \citep{gregg2003comparing}. Particularly, in the application of photovoltaics, the photoexcitation of \acp{OPV} results in excited electrons that are bound by the electric force to the positively charged holes left behind. The tightly-bound neutral electron-hole pairs, or \emph{excitons}, make charge extraction difficult in \acp{OPV}, resulting in low \ac{PCE}. As of 2019, the record \ac{PCE} for an \ac{OPV} cell is $17.4\%$ \citep{Meng2018,Cui2019}, while typical efficiencies are at $\sim 10\%$ \citep{NREL2019champion}. Inorganic photovoltaics consistently have \ac{PCE} $> 20\%$ \citep{NREL2019research}.

%, with the most recent record being $47.1\%$ \citep{Geisz2018}.

%On the other hand, light excitation of inorganic photovoltaics generate unbound electrons and holes, which can separate to the cathode and anode respectively.

Understanding the role of excitons is also important to the development of \ac{OLED} devices. Organic semiconductors emit energy in the form of light when excitons recombine---the excited electron ``falls" back into the hole. The colour and intensity of the emitted light is dependent on the energy and recombination rate of excitons formed in the material. The goal in light emitting applications is to maximize quantum efficiency, the number of photons emitted per charge carrier, while tuning the optical properties of the exciton. Understanding the underlying physics of exciton formation, dissociation, and recombination will not only optimize device performance, but contribute to the understanding of excited state phenomena in organic molecules.


% Organic semiconductors are a promising class of materials with a multitude of applications to electronics. 

% Organic semiconductor devices are lightweight, flexible, inexpensive, and easily fabricated when compared to conventional semiconductors \citep{lewis2007toward}. 

% However, in the application to photovoltaics, organic semiconductors are much less efficient than traditional photovoltaics made from silicon, due to the low dielectric constant of organic materials \citep{gregg2003comparing}. In order to optimise the power conversion of these devices, the electronic structures of organic photovoltaic molecules need to be understood at the nanoscale, using scanning probe techniques such as \ac{STM}, and \ac{STS} \citep{binnig1982surface}. 


%%%%%
\section{Excitons in organic semiconductors}

When looking at exciton physics in organic semiconductors, there are several energy scales we need to consider (\autoref{fig:intro:exciton}). The band gap is measured as the energy $E_{gap}$ from the \ac{HOMO} to the \ac{LUMO}, which are analogous to the valence and conductance bands inside a solid-state system, respectively.  Due to the low dielectric constant in organic materials, electric fields between charges are stronger due to lower screening. The attractive Coulomb force between the excited electron and the hole favours the formation of an exciton with a lower energy, giving an optical gap, $E_{opt}$, that differs from the HOMO-LUMO gap. The exciton binding energy $E_b$ is the difference between the band gap and the optical gap, and is typically on the order of $\sim \SI{1}{eV}$ for organic semiconductors \citep{knupfer2003exciton}. For excitons to dissociate, enough energy needs to be supplied to overcome the binding energy; thermal energy, which is on the order of $k_B T \sim \SI{0.1}{meV}$ ($k_B$ is the Boltzmann constant), is insufficient for thermal dissociation of excitons in organic materials.

% The energy of the \ac{HOMO} can be characterized by the ionization potential $E_I$, the energy required to excite an electron from the occupied orbital into the vacuum level, and \ac{LUMO} by the electron affinity $E_A$, the energy released by the addition of an electron from the vacuum level into the unoccupied orbital.

\begin{figure}[h]
    \centering
    \includegraphics[width=0.7\textwidth]{pictures/exciton_energy.png}
    \caption{Schematic of energy levels involved in organic molecule excitations. \textbf{(a)} A molecule in ground state. The HOMO and LUMO energies and the band gap of the molecule are labelled. \textbf{(b)} The optical gap, and exciton binding energy for an excited electron-hole pair are indicated for a typical exciton.}
    \label{fig:intro:exciton}
\end{figure}

Typically, \ac{OPV} systems are composed of two species of organic semiconducting materials: an electron donor and an electron acceptor molecule. In 1986, \textit{C.W. Tang} demonstrated the importance of the acceptor-donor heterojunction between organic semiconductors to the dissociation of excitons into free charge carriers \citep{tang1986two}. In \acp{OPV}, electrons inside an organic semiconductor are excited from \ac{HOMO} to \ac{LUMO} by photons that have energies greater than the band gap of the molecules. Excitons generated in the system act as quasi-particles that can diffuse through the material. At the interface, mismatched electron energy levels between the acceptor and donor generate an electric field that pulls apart the electron-hole pair, similar to the process at p-n junctions in inorganic photovoltaic systems. Acceptor organic molecules have high electron affinity, making it energetically favourable for the excited electron to transfer into the acceptor molecule \ac{LUMO} (\autoref{fig:intro:ct}). This delocalizes the exciton between two molecules, forming a \ac{CT} complex, and allows for exciton dissociation and charge generation in the system \citep{bernardo2014delocalization}.

\begin{figure}[h]
    \centering
    \includegraphics[width=0.8\textwidth]{pictures/CT_energy.png}
    \caption{Schemtic demonstrating the formation of the charge transfer state. \textbf{(1)} An incoming photon excites an electron in the donor system. The exciton forms due to Coulomb force between the electron and hole. \textbf{(2)} Before recombination occurs, the exciton diffuses to the heterojunction. The charge transfer exciton form. \textbf{(3)} With the exciton delocalized, dissociation occurs and the charge is transferred to the acceptor.}
    \label{fig:intro:ct}
\end{figure}

Excitons in \ac{OPV} materials typically have diffusion lengths of 1\SI{-30}{nm} \citep{proctor2013charge}, and have lifetimes ranging from attoseconds to microseconds \citep{tamai2015exciton}. There is a probability of exciton recombination, in which the excited electron drops in energy and re-emits the $E_{opt}$ as light, if the exciton is unable to diffuse to an interface to form the \ac{CT} state. Even after formation of the \ac{CT} state, or the charge-separated \ac{CT} state, recombination can still occur albeit with different probabilities and spectroscopic signatures \citep{deibel2010role}.

In light emission applications, the reverse of the photoexcitation process in \acp{OPV} is observed. A year after the two-layer \ac{OPV} publication, \emph{C.W. Tang and S.A. Van Slyke} created the first practical \ac{OLED} device \citep{Tang1987}. Using electron and hole transport layers, charges are directed from the cathode and anode into an emissive layer composed of organic semiconducting molecules, allowing for exciton formation and subsequent recombination. However, complications arise due to the nature of excitons in organic molecules. Vibration modes within the molecule create discrete energy states for each molecular orbital, as described in the Franck-Condon principle (\autoref{fig:intro:vib}). Relaxation between these discrete states can change the energy of the exciton, thereby changing the colour of emitted light.\footnote{This phenomenon is described by Kascha's rule. Differences in the absorption and emission spectra can arise, known as the Stoke's shift.} Additionally, excited triplet states can form in organic semiconductors, which generally have longer lifetimes and can decay non-radiatively, reducing the efficiency of \ac{OLED} devices \citep{kohler2009triplet}. 

\begin{figure}[H]
    \centering
    \includegraphics[width=0.6\textwidth]{pictures/franck_condon_transitions.png}
    \caption{Franck-Condon diagram. Vibrational modes are discrete quantum harmonic oscillator levels. \textbf{(1)} Photon excites an electron from singlet ground state into the third mode of the singlet first excited state, the $S_0(\nu=0) \rightarrow S_1(\nu'=3)$ absorption transition. \textbf{(2)} Relaxation in the excited molecule into the lowest vibrational mode $S_1(0)$, reducing the optical gap. \textbf{(3)} Exciton recombination into $S_0(2)$ due to the wavefunction overlap. This diagram demonstrates the $S_1(0) \rightarrow S_0(2)$ fluorescence transition.}
    \label{fig:intro:vib}
\end{figure}



%%%%%
\section{Scanning probe techniques}

% talk about what has been done, and why we should use spm
% Extensive research has been carried out on the optoelectronic properties of organic semiconductors, but often in the context of a device or a bulk ensemble of molecules. However, exciton formation, dissociation, and charge transfer happens on the molecular scale. 

\Acf{SPM} provides a suite of experimental characterization techniques that involve measuring the interaction, as a function of experiment parameters, between an extremely sharp probe and the sample of interest. The probe can then be raster scanned, using precise piezoelectric motors, across the sample to give a local spatial map of the interaction, or other determinable quantities. Unlike conventional optical microscopy techniques, \ac{SPM} resolution is determined by the sharpness of the probe which can be as small as a few picometres, allowing for real-space atomically resolved imaging that would not be possible with diffraction limited systems.

The \ac{SPM} techniques employed in this work include \ac{STM}, \ac{STS}, and \ac{STML}, which are all based on the tunnelling of electrons between the metallic tip and the sample. The samples investigated are organic semiconducting molecules supported by a conducting substrate, often with an insulating spacer layer. These three techniques can map out, with atomic resolution, the local structural, electronic, and optical properties of small assemblies of organic semiconducting molecules. The details of the techniques will be discussed in \autoref{ch:exptech}.

Extensive research on organic semiconducting molecules has been conducted, but often in the context of a device, or as a bulk ensemble of molecules. However, changes in heterojunction geometry, local electronic environment, and molecular structure can drastically affect the optoelectronic properties which occur at the nanometre length scale. \ac{SPM} can directly probe these effects at sub-molecular resolution, allowing for correlation between electron and exciton physics with changes in molecular configuration.

In all, six molecules were studied using \ac{SPM}. Prototypical organic semiconductors \acf{PTCDA} and \acf{ZnPc} were first studied to verify the setup used for \ac{STML}. PTCDA is a highly emissive dye molecule that is ideal for \ac{SPM} study as it is easily thermally deposited and highly planar in structure. A limited number of publications have demonstrated \ac{STML} on \ac{PTCDA} \citep{Cottin2018,Rzeznicka2011,Kimura2019}. Metal phthalocyanine molecules, such as ZnPc, are known emitters in \ac{STML} and have been studied extensively using \ac{SPM} \citep{Imada2016,Zhang2016,Doppagne2017,Miwa2019,cochrane2018molecularly,Kaiser2019,zhang2017sub,Cottin2018}. We further studied the effects of fluorination by looking at \ch{F8ZnPc}. Previous studies have shown that fluorination can tune the energy levels of the occupied and unoccupied states of ZnPc \citep{schwarze2016band}. 

Novel \ac{OLED} molecules were also studied; the molecules were derived from \acf{HMAT}, a highly planar electron donating molecule, which has a large photon absorption cross section and high quantum efficiency \citep{Paisley2020,Tonge2020,Chen2017}. These molecules include \acf{HMAT-O}, \acf{HMAT-TZ}, and \acf{HMAT-HZ}. Thin insulating films of NaCl were used to decouple the molecules from the noble metal surfaces used \citep{repp2005molecules}. However, the molecules may not be stable on the NaCl, depending on the molecule-substrate interactions present.

% In \ac{STM}, atomic resolution topography of the electronic landscape is obtained by measuring fluctations in the tunnelling current as the tip moves across our organic molecule. By sweeping the bias between the tip and the sample, we can identify resonances in the \ac{LDOS} spectroscopy as a function of energy corresponding to molecular orbitals. Performing this \ac{STS} measurement pixel-by-pixel, we can map out the spatial distribution of the occupied and unoccupied molecular states. Additionally, the tunnelling of charges between the tip and sample can generate local excitons in the organic semiconductor. In the \ac{STML} technique, these excitons can be optically detected when they recombine and emit an photon. This permits the mapping of the exciton formation, and, by spectroscopically resolving the emitted photons, the optical gap within the molecule. 

% talk about the disadvantages.
% There are disadvantages to using \ac{SPM} techniques. In particular, \ac{SPM} is highly dependent on the probe; the material and geometry of the probe can affect the measured interaction. The tip can be modified by indentation into a metallic surface, 

% This can affect the quality of topography, spectroscopy, and luminescence signal. Scanning tunnelling techniques also require that the molecules are deposited on a conducting substrate. Molecule-substrate interactions can obscure optoelectronic properties of the free molecule. 

% Additionally, molecule-substrate interaction can affect the 

% As it is difficult to characterize the tip \textit{in situ}, structural and electronic properties of the probe are often approximated with simple models.

% While \ac{SPM} offers high resolution local maps of the sample, the techniques are limited to imaging small select parts of the sample (on the order of $\sim$100 nm). Additionally, the scans are generally slower than other microscopy techniques, meaning that \ac{SPM} techniques generally do not have time resolution, and are susceptible to sample drift, mechanical vibration, and other sources of measurement uncertainty.


% talk about how they are used
% studies that have used SPM for organic molecules before













