%% The following is a directive for TeXShop to indicate the main file
%%!TEX root = diss.tex

\chapter{Introduction}
\label{ch:Introduction}

%%%%%
\section{Motivation}

% why are we interested in organic semiconductors
%   various applications, focusing specifically on OPV and OLED

% what is the interesting physics that we want to probe

% from EECE 531 final project
Organic semiconductors are a promising class of materials with a multitude of applications to electronics. Organic semiconductor devices are lightweight, flexible, inexpensive, and easily fabricated when compared to conventional semiconductors \citep{lewis2007toward}. However, in the application to photovoltaics, organic semiconductors are much less efficient than traditional photovoltaics made from silicon, due to the low dielectric constant of organic materials \citep{gregg2003comparing}. In order to optimise the power conversion of these devices, the electronic structures of organic photovoltaic molecules need to be understood at the nanoscale, using scanning probe techniques such as \ac{STM}, and \ac{STS} \citep{binnig1982surface}. 



%%%%%
\section{Scanning probe techniques}

% talk about how they are used
% studies that have used SPM for organic molecules before



%%%%%
\section{Acceptor/donor molecular systems}


%%%%%
\section{Organic light emitting molecules}








