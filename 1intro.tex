%% The following is a directive for TeXShop to indicate the main file
%%!TEX root = diss.tex

\chapter{Introduction}
\label{ch:Introduction}

%%%%%
\section{Motivation}

% why are we interested in organic semiconductors
%   various applications, focusing specifically on OPV and OLED

% what is the interesting physics that we want to probe

In the midst of the rapid development of silicon-based and inorganic semiconductors, starting from the early 1900s and have since led to age of modern computing and electronics, the first conducting organic polymer, polyacetylene, was synthesized in 1977 \citep{shirakawa1977synthesis,chiang1977electrical}. The discovery of non-insulating organic plastics established a new research area focused on the development of organic semiconductors, and led to the Chemistry Nobel Prize in 2000 awarded to their discoverers Heeger, MacDiarmid, and Shirakawa. After several decades of research, the class of semiconducting organic molecules have grown substantially, and organic semiconductors have found success in commercial applications such as \acp{OLED} \citep{zhu2011solution,ying2014white}, organic field effect transistors \citep{torsi2013organic,tatum2018pi}, and \acp{OPV} \citep{holliday2017recent,espinosa2015solution,gregg2003comparing,lewis2007toward}. 

The major advantage of organic semiconductors when compared to their inorganic counterparts is their relative ease of processing. Organic molecules can be thermally deposited or solution processed, allowing for cost-effective device fabrication \citep{yip2012recent,huang2019organic}. Additionally, by drawing from a suite of chemical synthesis techniques, new molecules can be created from Earth-abundant organic elements, and engineered with different functional groups to have the desired optical, electronic, and structural properties. This versatility allows for tunable light absorption and emission, transparency, and mechanical flexibility, which have already been applied to niche commercial products (\autoref{fig:intro:applications}). 

\begin{figure}
    \centering
    \includegraphics{}
    \caption{\FIXME{Add images of applications of organic semiconductors.}}
    \label{fig:intro:applications}
\end{figure}

Despite advantages in manufacturing and versatility, organic materials possess low electronic screening, a result of their intrinsically low dielectric constant (or low electric susceptibility), which presents challenges in optimising device performance \citep{gregg2003comparing}. Particularly, in the application of photovoltaics, the photoexcitation of \acp{OPV} results in negatively charged electrons that are bound by the electric force to the positively charge holes left behind. The tightly-bound neutral electron-hole pairs, or \emph{excitons}, make charge extraction difficult in \acp{OPV}, resulting in low \ac{PCE}. As of 2019, the record \ac{PCE} for an \acp{OPV} cell is $17.4\%$ \citep{Meng2018,Cui2019}, while typical efficiencies are at $\sim 10\%$ \citep{NREL2019champion}. Inorganic photovoltaics consistently have \ac{PCE} $> 20\%$ \citep{NREL2019research}, with the most recent record being $47.1\%$ \citep{Geisz2018}.

%On the other hand, light excitation of inorganic photovoltaics generate unbound electrons and holes, which can separate to the cathode and anode respectively.

Understanding the role of excitons is also important to the development of \ac{OLED} devices. Organic semiconductors emit energy in the form of light when excitons recombine---the excited electron ``falls" back into the hole. The colour and intensity of the emitted light is dependent on the energy and recombination rate of excitons formed in the material. In all applications of organic semiconductors, understanding exciton physics in organic materials with different electronic properties is key to the improvement of organic semiconductor devices.


% Organic semiconductors are a promising class of materials with a multitude of applications to electronics. 

% Organic semiconductor devices are lightweight, flexible, inexpensive, and easily fabricated when compared to conventional semiconductors \citep{lewis2007toward}. 

% However, in the application to photovoltaics, organic semiconductors are much less efficient than traditional photovoltaics made from silicon, due to the low dielectric constant of organic materials \citep{gregg2003comparing}. In order to optimise the power conversion of these devices, the electronic structures of organic photovoltaic molecules need to be understood at the nanoscale, using scanning probe techniques such as \ac{STM}, and \ac{STS} \citep{binnig1982surface}. 


%%%%%
\section{Excitons in organic semiconductors}

When looking at exciton physics in organic semiconductors, there are several energy scales we need to consider (\autoref{fig:intro:exciton}). 

In \acp{OPV}, electrons inside of an organic semiconductor are excited by photons that have energies greater than the band gap of the molecule. The band gap is measured as the energy $E_{gap}$ from the \ac{HOMO} to the \ac{LUMO}, which are analogous to the valence and conductance bands inside a solid-state system. The energy of the \ac{HOMO} can be characterized by the ionization potential $E_I$, the energy required to excite an electron from the occupied orbital into the vacuum level, and \ac{LUMO} by the electron affinity $E_A$, the energy released by the addition of an electron from the vacuum level into the unoccupied orbital.

Due to the low dielectric constant in organic materials, electric fields between charges are stronger due to lower screening. The attractive Coulomb force between the electron-hole pair favours the formation of an exciton with a lower energy known as the optical gap $E_{opt}$. The exciton binding energy $E_b$ is the difference between the band gap and the optical gap, and is typically on the order of $\sim \SI{1}{eV}$ for organic semiconductors \citep{knupfer2003exciton}. For excitons to dissociate, enough energy needs to be supplied to overcome the binding energy: thermal energy, which is on the order of $k_B T \sim \SI{0.1}{meV}$ ($k_B$ is the Boltzmann constant), is insufficient for thermal dissociation of excitons in organic photovoltaics.

\begin{figure}[h]
    \centering
    \includegraphics{}
    \caption{\FIXME{Add energy diagram of excitons.}}
    \label{fig:intro:exciton}
\end{figure}

Typically, \ac{OPV} systems are composed of two species of organic semiconducting materials: an electron donor and an electron acceptor molecule. In 1986, \textit{C.W. Tang} demonstrated the importance of the acceptor-donor heterojunction between organic semiconductors to the dissociation of excitons into free charge carriers, and hence the \ac{PCE} of \ac{OPV} devices \citep{tang1986two}. Excitons generated in the system act as quasi-particles, and can diffuse as a pair through the material. At the interface, mismatched electron energy levels between the acceptor and donor generate a high electric field that pulls apart the electron-hole pair (\autoref{fig:intro:ct}). Acceptor organic molecules have high electron affinity, making it energetically favourable for the excited electron to transfer into the acceptor molecule \ac{LUMO}. This delocalizes the exciton between two molecules, forming a charge transfer complex, and allows for exciton dissociation and electric current generation in the system \citep{bernardo2014delocalization}.

\begin{figure}[h]
    \centering
    \includegraphics{}
    \caption{\FIXME{Add energy diagram of CT state.}}
    \label{fig:intro:ct}
\end{figure}

Through the entire process, from exciton generation to dissociation, there is a probability of exciton recombination where the excited electron re-emits the energy as a photon and energetically ``falls" back into the hole. 






When an electron is excited in an organic semiconductor, whether by the absorption of a photon, in the case of photovoltaics, or injection of a charges, in the case of \ac{OLED} devices, excitons form rather than free charges due to the electric field generated by the local environment. In materials with a low dielectric constant, the attractive force from the hole along with the repulsive forces from neighbouring electrons are poorly screened, creating an electronic landscape that favours localisation of the excited electron. 

In \acp{OPV}, the goal is to separate the exciton into the free charges. Dissociation can occur in the presence of a high electric field, usually found a defects of interfaces between materials that have 



Excitons are formed when electrons are excited into a higher energy state but remain bound to the positively charged holes left behind. In materials that have poor electrical screening, neighbouring electrons and the hole creates an electric field that energetically favours the formation of an exciton. 


For a molecule in the ground state, the electrons occupy the lowest energy states, pairing off with two electrons with opposite spins in each orbital. 


%%%%%
\section{Scanning probe techniques}

% talk about how they are used
% studies that have used SPM for organic molecules before













