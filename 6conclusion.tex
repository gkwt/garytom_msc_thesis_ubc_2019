%% The following is a directive for TeXShop to indicate the main file
%%!TEX root = diss.tex

\chapter{Conclusion}
\label{ch:conc}

%%%%%

Molecular emission was successfully detected in our experiments, demonstrating that our system is capable of \ac{STML} experiments. In our studies of \ac{ZnPc}, broad molecular photoluminescence signals were observed at positive biases. The signals differ from previously reported \ac{STML} on the molecule at negative biases. As the molecules were more mobile when scanned at high positive bias, it is possible that the stability and motion of the molecules on the surface can affect the photon emission process. Other explanations include the presence of an alternate pathway for exciton generation, such as photon-mediated, or the tunnelling into higher energy unoccupied orbitals, such as LUMO+1. Signals detected in our study of \ac{PTCDA} are the first reported single molecule emission on uncharged \ac{PTCDA}. Similar to the emission detected on \ac{ZnPc}, the experiment was conducted at positive bias, and the detected emission peak was broad.

The effects of fluorination on the photoluminescence of ZnPc were studied through the \ac{SPM} of \ch{F8ZnPc}. \ac{STML} was detected at a bias of \SI{-2.5}{V}, similar to parameters of previously reported \ac{STML} on ZnPc. However, photon emission detected on \ch{F8ZnPc} varied in wavelength from \SI{630-640}{nm} for different \ch{F8ZnPc} on the surface. \ac{STM} scans at negative biases revealed a variety of topographically different molecules, and point \ac{STS} on the molecules showed differing \ac{LDOS} resonances for the molecules. Repeated sample preparation and replacing of molecules ruled out the presence of degraded molecule, and the switching between molecular species through tip manipulation seemed to indicate that the variety was due to changes in surface adsorption of the molecules. Molecules with smaller band gaps in the \ac{STS} spectra produced lower energy emission peaks in the \ac{STML} spectra, demonstrating correlation between the shifts in electronic states and the optical gap of formed excitons.

After a full system bake-out, a new sample was produced, and the variety of \ch{F8ZnPc} was observed again. However, the sample was noticeably different, with fewer defects on the NaCl bilayers, and different types of \ch{F8ZnPc} observed than those of the sample used for \ac{STML}. Regardless, the different types of molecules were categorized by \ac{STS} spectral features, \ac{STM} topographic differences, and relative positions on the NaCl lattice. We conclude that the presence of the fluorines in \ch{F8ZnPc} stabilize certain adsorption geometries not observed for ZnPc. These changes in conformation and interactions with the substrate can affect the electronic and optical properties of the molecules.

In our study of HMAT derivatives, using pixel-by-pixel \ac{STS}, sub-molecular images of the molecular orbitals for each HMAT molecule were generated. We find that the HOMO was localized around the donor complexes, the HMAT groups, of the molecules, while the LUMO was localized around the acceptor complexes. With the functionalization of increasingly electronegative acceptor groups, the band gaps obtained through \ac{STS} were observed to decrease, demonstrating the tuning of electron energy levels in molecules through chemical engineering. The results observed in our experiments qualitatively agree with DFT calculations of the HMAT derivatives in gas phase. \ac{STML} experiments were attempted for the HMAT derivatives, however, due to the large band gap, high biases would be required to generate excitons in the material, which could break or move the molecules on the surface.


% \section{Future directions}