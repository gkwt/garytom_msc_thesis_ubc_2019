%% The following is a directive for TeXShop to indicate the main file
%%!TEX root = diss.tex

\chapter{Luminescence from Organic Molecules}
\label{ch:opv}

Prior to the work done in this thesis, \ac{STML} from a single organic molecule has not yet been detected on our system. To optimize and ensure the viability of the experimental setup, \ac{STML} was performed on prototypical organic semiconducting molecules PTCDA \citep{Rzeznicka2011, Kimura2019} and ZnPc \citep{Zhang2016, Doppagne2017, Zhang2017, Imada2016, Doppagne2018, Miwa2019}, both of which have had successful reports of \ac{STML} experiments. Fluorination of phthalocyanine molecules have previously been demonstrated as a method to tune the electronic and optical energies of the molecule in bulk \citep{schwarze2016band, warren2019controlling}. Further experiments were carried out on \ch{F8ZnPc} to explore the effects of fluorination on the structural, electronic and optical properties of ZnPc.


\section{Plasmon emission from substrates}

Plasmon emission from the bare metallic surface can vary in energy and intensity depending on the geometry of the tip. Aside from attaining a sharp metallic tip for \ac{STM} and \ac{STS}, the tip needs to be poked and pulsed until it produce a strong plasmonic response in the energy region of interest. Representative spectra of the plasmon on Ag(111) and Au(111) with Ag tip is presented in \autoref{fig:opv:metal-plasmon}. The same Ag tip was used to acquire both spectra, allowing for comparison. The plasmon emission on the Au(111) is weaker in intensity and lower in energy than the emission on Ag(111). The lower intensity was likely because of the change in tip-sample position due to the lower height of the thin film, resulting in a drop of about an order of mangitude in intensity. The shift in photon energy was caused by the inherent dielectric response of the materials \citep{olmon2012optical, yang2015optical}.

% The shift in photon energy was because of the dielectric function of the two materials: gold with a minimal imaginary dielectric function at \SI{1.8}{eV} or \SI{680}{nm} \citep{olmon2012optical}, and silver at \SI{3.2}{eV} or \SI{390}{nm} \cite{yang2015optical}.

\begin{figure} [h]
    \centering
    %\includegraphics[width=3in]{file}
    \caption{\FIXME{Ag and Au plasmons, spectra taken with the same tip.}}
    \label{fig:opv:metal-plasmon}
\end{figure}



When \ac{STML} experiments are carried out on molecular species, they are decoupled by a bilayer (2\ac{ML}) NaCl, which can modify the plasmon emission. Using the same Ag tip, modification of the plasmon emission on Ag(111) and Au(111) by layers of NaCl can be seen in \autoref{fig:opv:nacl-plasmon}. Overall, there is an enhancement and redshift in the detected photons when compared to the bare metal substrate.


\begin{figure} [h]
    \centering
    %\includegraphics[width=3in]{file}
    \caption{\FIXME{Ag and Au plasmons, modified by nacl (2ML and 3ML?)}}
    \label{fig:opv:nacl-plasmon}
\end{figure}

Thicker layers of NaCl further dissociate the molecule from the metallic substrate, allowing for a more accurate study of the intrinsic properties of the molecule \citep{repp2005molecules}. Thicker layers of NaCl have demonstrated enhanced molecular luminescence; the effect of a smaller tip-sample junction and more effective dissociation of the molecule from the metal \citep{Zhang2017,Kroger2018}. However, a closer tip results in stronger interactions between the tip and the molecule, affecting the stability of the molecule on the NaCl. Bilayer NaCl gave the most consistent and stable configurations for \ac{STML}.



\section{ of ZnPc and PTCDA}








\section{\ac{STML} of fluorinated \ch{ZnPc}}






\subsection{Effects of adsorption geometry}





%%%%%
