%% The following is a directive for TeXShop to indicate the main file
%%!TEX root = diss.tex

%% https://www.grad.ubc.ca/current-students/dissertation-thesis-preparation/preliminary-pages
%% 
%% LAY SUMMARY Effective May 2017, all theses and dissertations must
%% include a lay summary.  The lay or public summary explains the key
%% goals and contributions of the research/scholarly work in terms that
%% can be understood by the general public. It must not exceed 150
%% words in length.

\chapter{Lay Summary}

Solar energy can be converted into electrical power using photovoltaic cells. However, most photovoltaic cells today are silicon-based and are difficult to manufacture, making them expensive to implement. Organic photovoltaics have emerged as inexpensive and lightweight alternatives due to their relative ease of fabrication. However, electrons excited by light in organic materials are tightly bound to the holes that they left behind. These bound electrons can fall back into the holes in a process called recombination, reducing the efficiency of the device. This issue can be mitigated by creating interfaces of different organic molecules, making an electronic enviornment that pulls apart the electron-hole pairs. In the described research, I use scanning probe techniques to examine the electronic and optical properties of such interfaces to better understand the charge transfer mechanisms. Research in this field will contribute to making efficient organic photovoltaic cells, and provide fundamental knowledge to quantum physics.

\endinput
With an increasing global demand for energy, along with a dwindling supply of fossil fuels, there is a heightened interest in the research of sustainable energy sources, in particular, the abundant and accessible solar energy. 

Incoming light energy from the sun excites electrons in photovoltaic materials, which in turn produces a current that can be used for electrical work. 